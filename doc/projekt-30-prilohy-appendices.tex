% Tento soubor nahraďte vlastním souborem s přílohami (nadpisy níže jsou pouze pro příklad)
% This file should be replaced with your file with an appendices (headings below are examples only)

% Umístění obsahu paměťového média do příloh je vhodné konzultovat s vedoucím
% Placing of table of contents of the memory media here should be consulted with a supervisor
\chapter{Obsah priloženého pamäťového média}

\chapter{Návod na inštaláciu}
Program pri svojej činnosti využíva knižnicu ImageMagick vo verzii 7.0.5-4. Aktuálnu verziu tejto knižnice si môžete stiahnuť na webovej stránke \url{www.imagemagick.org} v~sekcii download\footnote{ImageMagick \url{https://www.imagemagick.org/script/download.php}}.

Nainštalovanú verziu knižnice ImageMagick zistíte príkazom:
\begin{center}
\begin{tabular}{c}
\begin{lstlisting}
identify -version
\end{lstlisting}
\end{tabular}
\end{center}

Pred kompiláciou programu bude možno potrebné nastaviť premennú prostredia\\ \verb|PKG_CONFIG_PATH| príkazom
\begin{center}
\begin{tabular}{c}
\begin{lstlisting}
export PKG_CONFIG_PATH=/usr/local/lib/pkgconfig
\end{lstlisting}
\end{tabular}
\end{center}

Následne je možné skompilovať program pomocou príkazu {\tt make}, ktorý vytvorí spustiteľný program s~názvom {\tt watermarking} a priečinok {\tt attacks} do ktorého budú uložené obrázky po prevedení útokov pomocou prepínača \verb|--attack|. Príklad spustenia programu
\begin{center}
\begin{tabular}{c}
\begin{lstlisting}
./watermarking --input lena.jpg --wm-input fit.pbm
\end{lstlisting}
\end{tabular}
\end{center}
